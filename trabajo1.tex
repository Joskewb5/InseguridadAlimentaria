\documentclass[Royal,times,sageh]{sagej}

\usepackage{moreverb,url,natbib, multirow, tabularx}
\usepackage[colorlinks,bookmarksopen,bookmarksnumbered,citecolor=red,urlcolor=red]{hyperref}



% tightlist command for lists without linebreak
\providecommand{\tightlist}{%
  \setlength{\itemsep}{0pt}\setlength{\parskip}{0pt}}

% From pandoc table feature
\usepackage{longtable,booktabs,array}
\usepackage{calc} % for calculating minipage widths
% Correct order of tables after \paragraph or \subparagraph
\usepackage{etoolbox}
\makeatletter
\patchcmd\longtable{\par}{\if@noskipsec\mbox{}\fi\par}{}{}
\makeatother
% Allow footnotes in longtable head/foot
\IfFileExists{footnotehyper.sty}{\usepackage{footnotehyper}}{\usepackage{footnote}}
\makesavenoteenv{longtable}




\begin{document}


\setcitestyle{aysep={,}}

\title{Predicción de la inseguridad alimentaria leve en Bolivia a partir
de características sociodemográficas y del hogar}

\runninghead{Josias Sirpa}

\author{Josias Milan Sirpa Pinto*\affilnum{1}}

\affiliation{\affilnum{1}{Universidad Católica Boliviana San Pablo sede
``La Paz''}}

\corrauth{Josias Sirpa, Carrera de Economía e Inteligencia de Negocios,
UCB, La Paz, Bolivia}

\email{\href{mailto:josias.sirpa@ucb.edu.bo}{\nolinkurl{josias.sirpa@ucb.edu.bo}}}

\begin{abstract}
Este estudio aplica técnicas de clasificación supervisada para predecir
la inseguridad alimentaria leve en hogares bolivianos, utilizando datos
de la Encuesta de Hogares 2023. Se construyó una variable binaria que
identifica la presencia de inseguridad alimentaria leve. A partir de
variables sociodemográficas y de condiciones de vivienda del jefe/a de
hogar, se entrenaron y compararon distintos modelos: logit, probit,
Naive Bayes, KNN, árboles de decisión (CART y C5.0) y redes neuronales.
Los resultados muestran que el modelo logit obtuvo el mejor desempeño en
términos de exactitud, con un 63.9\%, seguido por redes neuronales
(63.0\%). Estos modelos permiten identificar perfiles de riesgo y
aportar evidencia para el diseño de políticas públicas orientadas a
combatir la inseguridad alimentaria en el país. .
\end{abstract}

\keywords{Inseguridad Alimentaria; Bolivia; Predicción; Logit;}

\maketitle

\newpage

\section{Introducción}\label{introducciuxf3n}

La inseguridad alimentaria es una problemática persistente que afecta a
millones de personas en todo el mundo, especialmente en países en
desarrollo. En Bolivia, a pesar de la ampliación de la cobertura de
programas sociales, un número importante de hogares continúa enfrentando
dificultades para acceder de forma regular y suficiente a alimentos
nutritivos y adecuados. Este fenómeno, además de constituir una
vulneración de derechos fundamentales, tiene implicancias directas en la
salud, el desarrollo infantil, la productividad y el bienestar general
de la población.

Ante esta situación, resulta necesario contar con herramientas
analíticas que permitan identificar de manera temprana los hogares en
riesgo de sufrir inseguridad alimentaria, que muchas veces pasa
desapercibida en los programas de asistencia. El uso de métodos de
minería de datos y aprendizaje automático brinda una oportunidad valiosa
para construir modelos predictivos que, a partir de características
observables de los hogares y sus miembros (especificamente jefes/as de
hogar), permitan anticipar escenarios de vulnerabilidad alimentaria.

\section{Objetivos}\label{objetivos}

\subsection{Objetivo General}\label{objetivo-general}

Desarrollar un modelo de clasificación que permita predecir la presencia
de inseguridad alimentaria en hogares bolivianos, a partir de variables
sociodemográficas de los jefes/as de hogar y características de la
vivienda, utilizando la Encuesta de Hogares 2023 como base de análisis.

\subsection{Objetivos Específicos}\label{objetivos-especuxedficos}

\begin{itemize}
\item
  Identificar y seleccionar las variables independientes relevantes que
  influyen en la presencia de inseguridad alimentaria en los hogares.
\item
  Construir y entrenar distintos modelos de clasificación supervisada
  (logit, probit, Naive Bayes, KNN, CART, C5.0 y redes neuronales) para
  evaluar su desempeño predictivo.
\item
  Comparar el desempeño de los modelos mediante métricas como la
  exactitud (accuracy) y la matriz de confusión, seleccionando el modelo
  con mejor capacidad de clasificación.
\item
  Diseñar una función que permita aplicar el modelo final a nuevos
  casos, facilitando la predicción de inseguridad alimentaria leve con
  base en las características ingresadas.
\end{itemize}

\section{Motivación´}\label{motivaciuxf3n}

El presente trabajo surge como una propuesta para aplicar técnicas de
clasificación al análisis de la inseguridad alimentaria en Bolivia, con
base en la Encuesta de Hogares 2023. La motivación principal es
desarrollar un modelo que no solo clasifique con precisión, sino que
también sea interpretable y aplicable en contextos de política pública.
Si los organismos estatales, ONG o instancias locales pudieran contar
con modelos predictivos eficientes, sería posible focalizar con mayor
efectividad los esfuerzos de prevención y mitigación de la inseguridad
alimentaria, optimizando los recursos disponibles.

\section{Marco Teórico}\label{marco-teuxf3rico}

La seguridad alimentaria existe cuando todas las personas, en todo
momento, tienen acceso físico, social y económico a alimentos
suficientes, inocuos y nutritivos, que satisfacen sus necesidades
alimentarias y preferencias, para llevar una vida activa y saludable
(\citet{fao2011}). Este concepto, ampliamente aceptado en el ámbito
internacional, se basa en cuatro dimensiones fundamentales que permiten
su evaluación integral:

\begin{itemize}
\item
  \textbf{Disponibilidad}: La disponibilidad física se refiere a la
  oferta de alimentos, tanto a nivel nacional como local. Esta dimensión
  depende de factores como la producción agrícola interna, los niveles
  de reservas, las importaciones y exportaciones, y el comercio neto. Es
  la dimensión más asociada a la oferta de alimentos y a la capacidad
  del sistema alimentario de abastecer a la población.
\item
  \textbf{Acceso}: El acceso económico y físico a los alimentos implica
  que las personas no solo deben poder encontrar alimentos en el mercado
  o en sus entornos, sino también tener los medios para adquirirlos.
  Esto incluye el ingreso del hogar, la estabilidad de precios, la
  infraestructura física (caminos, transporte) y los mecanismos sociales
  que facilitan el acceso equitativo. Una adecuada disponibilidad no
  garantiza la seguridad alimentaria si las personas no tienen capacidad
  adquisitiva.
\item
  \textbf{Utilización}: La utilización biológica hace referencia al uso
  adecuado de los alimentos por parte del organismo. Esta dimensión
  abarca aspectos como la preparación, diversidad y calidad de la dieta,
  las prácticas de higiene y salud, y la absorción eficiente de
  nutrientes. La seguridad alimentaria implica, por tanto, que los
  alimentos no solo estén disponibles y sean accesibles, sino también
  que su consumo conlleve beneficios nutricionales efectivos para los
  individuos.
\item
  \textbf{Estabilidad}: La estabilidad implica que las otras tres
  dimensiones (disponibilidad, acceso y utilización) se mantengan a lo
  largo del tiempo. Una persona puede tener acceso adecuado a los
  alimentos en un momento dado, pero aún así estar en situación de
  inseguridad alimentaria si ese acceso no está garantizado de manera
  sostenida. Factores como el desempleo, los conflictos, las crisis
  económicas, las variaciones estacionales o los desastres naturales
  pueden alterar esta estabilidad.
\end{itemize}

Estas dimensiones no deben analizarse de forma aislada, ya que solo una
combinación armónica entre ellas permite alcanzar una seguridad
alimentaria sostenible. Este enfoque ha sido clave para el diseño de
políticas públicas y estrategias integrales de combate al hambre,
especialmente en contextos vulnerables como el boliviano.

\subsection{Medición basada en experiencia del
hogar}\label{mediciuxf3n-basada-en-experiencia-del-hogar}

La \citet{fao2024} utiliza escalas validadas como la ELCSA (Escala
Latinoamericana y Caribeña de Seguridad Alimentaria) y el FIES (Food
Insecurity Experience Scale). Estas herramientas capturan directamente
la vivencia de los hogares (preocupación, reducción de calidad,
cantidad, hambre), distinguiendo niveles de severidad: leve, moderado y
severo. En Bolivia la Encuesta Hogares de \citet{ine2023} utiliza la
escala FIES, por lo cual a continuación desarrollaremos la misma.

\subsection{Food Insecurity Experience Scale
(FIES)}\label{food-insecurity-experience-scale-fies}

La Escala de Experiencia de Inseguridad Alimentaria (FIES) es una
herramienta desarrollada por la FAO en el marco del proyecto Voices of
the Hungry, con el objetivo de medir la inseguridad alimentaria desde la
vivencia directa de las personas. A diferencia de otros enfoques
tradicionales, como la Prevalencia de la Subalimentación, que evalúan el
fenómeno de forma indirecta a partir de la disponibilidad de alimentos o
los niveles de ingreso, la FIES captura experiencias concretas
vinculadas a dificultades en el acceso a alimentos por falta de
recursos. Este enfoque centrado en la experiencia permite aproximarse al
componente subjetivo y cotidiano de la inseguridad alimentaria,
reflejando con mayor precisión los impactos reales sobre los hogares.

La escala consiste en un módulo de ocho preguntas que indagan sobre
experiencias relacionadas con preocupaciones por el acceso a alimentos,
reducción en la calidad o cantidad de la dieta, e incluso episodios de
hambre. Estas preguntas tienen respuestas dicotómicas (sí/no) y se
administran con un período de 12 meses, sin embargo, esto puede variar
dependiendo del diseño de la encuesta. En Bolivia, la Encuesta de
Hogares 2023 aplica la versión para hogares, en la cual una persona
responde en representación del grupo familiar.

Las ocho preguntas son las siguientes:

\begin{enumerate}
\def\labelenumi{\arabic{enumi}.}
\tightlist
\item
  Se haya preocupado por no tener suficientes alimentos para comer
\item
  No haya podido comer alimentos sanos o nutritivos
\item
  Haya comido poca variedad de alimentos
\item
  Haya tenido que saltarse una comida
\item
  Haya comido menos de lo que pensaba que debía comer
\item
  Su hogar se haya quedado sin alimentos
\item
  Haya sentido hambre pero no comió
\item
  Haya dejado de comer durante todo un día
\end{enumerate}

La FIES se construye como una escala estadística basada en la teoría de
respuesta al ítem (TRI), empleando modelos probabilísticos que permiten
clasificar a los hogares en distintos niveles de severidad de la
inseguridad alimentaria (leve, moderada o severa). Esta metodología,
similar a la utilizada en pruebas psicométricas, asegura la
comparabilidad de los resultados a nivel internacional, lo que convierte
a la FIES en uno de los dos indicadores oficiales para monitorear el
Objetivo de Desarrollo Sostenible 2.1 \citet{unods2025}, que busca poner
fin al hambre y garantizar el acceso a una alimentación adecuada.

Entre sus principales fortalezas se destaca su capacidad para producir
estimaciones comparables entre países y subpoblaciones, así como su
aplicabilidad en diversos contextos socioculturales. Además, su
integración en encuestas nacionales permite identificar grupos
vulnerables, monitorear políticas públicas y explorar factores de riesgo
asociados a la inseguridad alimentaria. No obstante, su análisis
requiere cierto nivel de especialización técnica y el uso de software
específico, lo cual puede representar un desafío para algunas
instituciones. La FAO, sin embargo, ofrece herramientas estadísticas,
módulos en varios idiomas y asistencia técnica para facilitar su
implementación.

\subsection{Contexto boliviano}\label{contexto-boliviano}

Bolivia enfrenta una situación compleja en materia de seguridad
alimentaria, determinada por factores estructurales como la pobreza, la
desigualdad y la alta vulnerabilidad territorial. Según
\citet{mallea2014}, el 37\% de la población boliviana vive en extrema
pobreza, una condición que repercute directamente en el acceso y la
calidad de los alimentos. Esta problemática se refleja también en
indicadores preocupantes como la desnutrición crónica infantil, que
alcanza al 32\% de la población menor de cinco años, y en una alarmante
incidencia de anemia que afecta a ocho de cada diez niños entre los 6 y
23 meses de edad.

El índice de desarrollo humano del país se encuentra en un nivel
medio-bajo (0,729), y aproximadamente el 53\% de los municipios
presentan un grado alto o muy alto de vulnerabilidad a la inseguridad
alimentaria. Las regiones rurales, especialmente el altiplano,
evidencian las peores condiciones de ingesta calórica y proteica,
exacerbadas por factores como el aislamiento geográfico, el bajo acceso
a servicios básicos y la escasa capacidad de respuesta ante eventos
climáticos adversos (\citet{mallea2014}).

La población indígena, que representa alrededor del 65\% del total
nacional, es particularmente afectada por estas condiciones, debido a
marcadas desigualdades en el acceso a recursos, tecnologías y
oportunidades. Además, la migración campo-ciudad ha generado nuevas
formas de inseguridad alimentaria en áreas periurbanas, donde se
incrementa la demanda de servicios sin que existan mecanismos de
protección adecuados.

En respuesta a esta realidad, el Estado boliviano ha incorporado
instrumentos como la FIES en encuestas oficiales, lo que permite contar
con una base empírica más sólida para evaluar y monitorear la
inseguridad alimentaria. Sin embargo, como advierte la literatura, la
información nacional sigue siendo escasa, dispersa y metodológicamente
limitada, lo que dificulta una planificación eficaz
(\citet{mallea2014}).

El presente estudio cobra especial relevancia al proponer una
herramienta predictiva basada en minería de datos, que permite anticipar
situaciones de inseguridad alimentaria a partir de características
observables del hogar. Esto no solo complementa los esfuerzos
existentes, sino que también contribuye al diseño de políticas públicas
más focalizadas, orientadas a reducir la vulnerabilidad y mejorar la
calidad de vida de las poblaciones más afectadas.

\section{Descripción del dataset ´}\label{descripciuxf3n-del-dataset}

La presente investigación utiliza los microdatos de la Encuesta de
Hogares 2023 (EH2023), elaborada por el Instituto Nacional de
Estadística de Bolivia (INE). Esta encuesta tiene representatividad
nacional, departamental y por área geográfica (urbana y rural), y recoge
información detallada sobre las condiciones sociodemográficas,
económicas, educativas y habitacionales de los hogares bolivianos.

En particular, el módulo de seguridad alimentaria de la EH2023 incluye
preguntas asociadas a la Escala de Experiencia de Inseguridad
Alimentaria (FIES), desarrollada por la FAO. La escala evalúa la
inseguridad alimentaria en tres niveles de severidad: leve, moderada y
severa. Para este estudio, se optó por analizar la variable s07a\_01,
que corresponde a la primera pregunta del módulo FIES: ``¿Usted o algún
miembro de su hogar estuvo preocupado porque no tenían suficientes
alimentos?''. Esta pregunta mide la inseguridad alimentaria leve,
centrada en la dimensión perceptiva de la preocupación, y fue
seleccionada por su alta cobertura y baja incidencia de valores
perdidos, a diferencia de las preguntas asociadas a las categorías
moderada y severa, que presentaban menor variabilidad y mayor sesgo en
su distribución.

\includegraphics[width=1\linewidth]{trabajo1_files/figure-latex/i1-1}

\begin{center}
\textbf{Figura 1.} Escala FIES.
\end{center}

En la figura 1 podemos ver como la FIES clasifica las respuestas de las
ocho preguntas de la encuesta, para medir una inseguridad leve, moderada
y grave.

La variable dependiente fue transformada en binaria con los siguientes
criterios:

\begin{itemize}
\item
  Se codificó como 1 si el hogar respondió afirmativamente (``sí'') a la
  pregunta de preocupación por la disponibilidad de alimentos.
\item
  Se codificó como 0 si la respuesta fue negativa (``no'') o ``no
  sabe/no responde''. Estas últimas se agruparon con las respuestas
  negativas por su baja frecuencia y para mantener una dicotomía
  estadísticamente robusta.
\end{itemize}

Se seleccionaron variables explicativas provenientes de distintos
módulos del cuestionario de hogares:

\begin{itemize}
\item
  Características del jefe/a de hogar (eh23p): sexo, edad,
  autoidentificación indígena, es decir, si pertenece a algún pueblo
  indígena (p\_indig) y años de estudio (aestudio), finalmente la base
  de datos se convirtio en eh23p\_f
\item
  Condiciones de la vivienda (eh23v): número total de personas en el
  hogar (totper), ingreso total del hogar (yhog), tipo de vivienda, es
  decir, si es propia, rentada, etc. (s06a\_02), material del piso
  (s06a\_06), disponibilidad de un cuarto de cocina (s06a\_14) y acceso
  a internet (s06a\_19), la base de datos final es eh23v\_f
\item
  Disponibilidad de bienes durables (eh23e): tenencia de refrigerador
  (refrigerador), transformada a variable binaria (1 = posee, 0 = no
  posee), inicialmente tambien se considero tenencia de cocina en una
  variable binaria, sin embargo, al realizar los modelos logit demostro
  no ser significante por lo cual se elimino. Además se alargo la base
  de datos, ya que originalmente estaba clasificada por items (del 1 al
  17), se filtro refrigerador y cocina (items 2 y 4) y se transformo a
  variables binarias, la base de datos final es eh23ef.
\end{itemize}

Para asegurar la correcta integración de los módulos, se utilizó el
identificador único de hogar (folio). La base de datos fue depurada para
eliminar observaciones con valores perdidos (sin embargo no existian).
Posteriormente, se realizó una partición aleatoria de la muestra en
conjuntos de entrenamiento (70\%) y prueba (30\%), manteniendo la
distribución de la variable dependiente para garantizar un entrenamiento
balanceado.

En resumen, se utilizo las siguientes variables independientes para el
ejercicio (no se incluye variable dependiente s07a\_01):

\begin{longtable}[]{@{}
  >{\raggedright\arraybackslash}p{(\columnwidth - 4\tabcolsep) * \real{0.1781}}
  >{\raggedright\arraybackslash}p{(\columnwidth - 4\tabcolsep) * \real{0.2329}}
  >{\raggedright\arraybackslash}p{(\columnwidth - 4\tabcolsep) * \real{0.5890}}@{}}
\caption{Variables utilizadas en el análisis, clasificadas según su
origen}\tabularnewline
\toprule\noalign{}
\begin{minipage}[b]{\linewidth}\raggedright
Variable
\end{minipage} & \begin{minipage}[b]{\linewidth}\raggedright
Pertenece.a
\end{minipage} & \begin{minipage}[b]{\linewidth}\raggedright
Detalles
\end{minipage} \\
\midrule\noalign{}
\endfirsthead
\toprule\noalign{}
\begin{minipage}[b]{\linewidth}\raggedright
Variable
\end{minipage} & \begin{minipage}[b]{\linewidth}\raggedright
Pertenece.a
\end{minipage} & \begin{minipage}[b]{\linewidth}\raggedright
Detalles
\end{minipage} \\
\midrule\noalign{}
\endhead
\bottomrule\noalign{}
\endlastfoot
folio & Vivienda & Identificador único del hogar \\
sexo & Jefe/a del hogar & Sexo del jefe/a del hogar \\
edad & Jefe/a del hogar & Edad del jefe/a del hogar (años cumplidos) \\
p\_indig & Jefe/a del hogar & Pertenencia a algún pueblo indígena \\
aestudio & Jefe/a del hogar & Años de estudio del jefe/a del hogar \\
yhog & Vivienda & Ingreso mensual del hogar en bolivianos \\
totper & Vivienda & Total de personas en el hogar \\
s06a\_02 & Vivienda & Tipo de tenencia de la vivienda \\
s06a\_06 & Vivienda & Material predominante del piso \\
s06a\_14 & Vivienda & Disponibilidad de cocina \\
s06a\_19 & Vivienda & Acceso a internet \\
refrigerador & Vivienda & Disponibilidad de refrigerador \\
\end{longtable}

En total, la base final de análisis incluyó 12.815 hogares, con un
conjunto de predictores representativos de las condiciones
estructurales, materiales y sociodemográficas que, desde la literatura
especializada, se asocian a la probabilidad de experimentar inseguridad
alimentaria leve.

\section{Metodología}\label{metodologuxeda}

La presente investigación aplica métodos de clasificación supervisada
para predecir la presencia de inseguridad alimentaria leve en hogares
bolivianos. Se parte de un enfoque de analítica predictiva, en el cual
se busca estimar la probabilidad de que un hogar experimente inseguridad
alimentaria leve a partir de un conjunto de variables explicativas
observadas. Para ello, se entrenaron y compararon varios modelos de
clasificación, utilizando una partición aleatoria de los datos en
subconjuntos de entrenamiento (70\%) y prueba (30\%).

En términos formales, se busca estimar una función f tal que:

\[ y = f(x) \]

donde y es una variable binaria con valores de 0 para ``NO'' y 1 para
``SI'', y representa la variable dependiente (inseguridad alimentaria
leve) y x es el vector de predictores. El conjunto de modelos utilizados
incluye:

\begin{itemize}
\tightlist
\item
  Regresión logística (logit)
\item
  Regresión probit
\item
  Naive Bayes
\item
  K-nearest neighbors (KNN)
\item
  Árbol de clasificación CART
\item
  Árbol C5.0
\item
  Redes neuronales artificiales (ANN)
\end{itemize}

\subsection{Regresión logística y
probit}\label{regresiuxf3n-loguxedstica-y-probit}

Estos modelos estiman la probabilidad de que un hogar pertenezca a la
categoría con inseguridad alimentaria:

\[
P(x) = \mathbb{P}(Y=1|X=x) = \mathbb{E}[Y|X=x]
\]

\textbf{Modelo Probit:} \[
P(x) = \Phi(x'\beta)
\]

\textbf{Modelo Logit:} \[
P(x) = \frac{1}{1 + \exp(-x'\beta)}
\]

donde \(\phi\) es la función de distribución acumulada normal estándar.
Ambos modelos fueron ajustados mediante glm() en R, utilizando la
función step() para seleccionar variables significativas.

\subsection{Naive Bayes}\label{naive-bayes}

El clasificador Naive Bayes se basa en el teorema de Bayes, con el
supuesto fuerte de independencia condicional entre predictores:

\[
P(Y|X) = \frac{P(Y) \cdot P(X|Y)}{P(X)} \approx P(Y) \cdot P(X|Y)
\]

\text{Con independencia condicional:}

\[
P(Y|X) \propto P(Y) \cdot \prod_{i=1}^{p} P(X_i|Y)
\]

Se utilizó la librería naivebayes, evaluando el rendimiento mediante la
matriz de confusión sobre la base de prueba.

\subsection{K-Nearest Neighbors (KNN)}\label{k-nearest-neighbors-knn}

El método KNN asigna una observación nueva a la clase más frecuente
entre sus k vecinos más cercanos en el espacio de covariables. La
distancia empleada fue la euclideana:

\text{Distancia Euclideana:}

\[
d(x, x_i) = \sqrt{ \sum_{j=1}^{p} (x_j - x_{ij})^2 }
\]

El valor de k se eligió como la raíz cuadrada del número de
observaciones en la base de entrenamiento. Se normalizaron las variables
predictoras para evitar sesgos por escala.

\subsection{Árboles de clasificación: CART y
C5.0}\label{uxe1rboles-de-clasificaciuxf3n-cart-y-c5.0}

Los árboles de decisión construyen reglas de partición mediante medidas
de impureza como la entropía o el índice Gini:

\text{Entropía:}

\[
\text{Entropía}(t) = - \sum_{i=1}^{k} p_i \log_2(p_i)
\]

\text{Índice Gini:}

\[
\text{Gini}(t) = 1 - \sum_{i=1}^{k} p_i^2
\]

El algoritmo CART se implementó mediante la librería rpart, mientras que
C5.0 se ajustó con C50. Ambos modelos generan una estructura de árbol
que facilita la interpretación de las reglas de decisión.

\subsection{Redes neuronales artificiales
(ANN)}\label{redes-neuronales-artificiales-ann}

Las redes neuronales permiten modelar relaciones no lineales entre las
variables explicativas y la respuesta. Una ANN se compone de capas de
nodos (entrada, ocultas y salida) interconectadas por pesos ajustables.
La salida de una red simple con una capa oculta se puede representar
como:

\text{Red neuronal con una capa oculta:}

\[
\hat{y}(x) = f\left( \sum_{j=1}^{h} w_j \cdot \sigma\left( \sum_{i=1}^{p} w_{ij} x_i + b_j \right) + b_o \right)
\]

\text{Función de activación sigmoide:}

\[
\sigma(z) = \frac{1}{1 + e^{-z}}
\]

donde \(\sigma\) es una función de activación (e.g.~sigmoide), y los
pesos w se ajustan mediante backpropagation y descenso del gradiente. Se
empleó la librería nnet con una arquitectura de una capa oculta y hasta
20 nodos.

\subsection{Evaluación del
rendimiento}\label{evaluaciuxf3n-del-rendimiento}

El desempeño de cada modelo se evaluó mediante la matriz de confusión y
la métrica de exactitud (accuracy) sobre la base de prueba. También se
examinó la sensibilidad y especificidad para interpretar la capacidad de
clasificación en función de verdaderos positivos y negativos.

\section{Resultados, análisis y
predicción}\label{resultados-anuxe1lisis-y-predicciuxf3n}

Al inicio del ejercicio se consideraron un total de 17 variables (sin
considerar folio y la variable dependiente) que se consideraban
significativas, sin embargo, al momento de realizar el logit y gracias a
la función step(), se termino descartando las siguientes variables:

\begin{itemize}
\tightlist
\item
  s06a\_12: Existencia de energia electrica en la vivienda.
\item
  leer\_esc: Si el jefe/a de hogar sabia leer y escribir.\\
\item
  s06a\_08a: Dias a la semana con agua en la vivienda.
\item
  cocina: Variable binaria de tenencia de cocina en la vivienda
\item
  s06a\_07: Proveniencia del agua de la vivienda (pozo, alcantarilla,
  etc.)
\item
  s06a\_09: Tipo de baño en la vivienda
\end{itemize}

A continuación se muestran los resultados de los diferentes métodos y al
final un resumen

\subsection{Logit}\label{logit}

\begin{longtable}[]{@{}lr@{}}
\caption{Resultados del modelo Logit (Matriz de
confusión)}\tabularnewline
\toprule\noalign{}
Métrica & Valor \\
\midrule\noalign{}
\endfirsthead
\toprule\noalign{}
Métrica & Valor \\
\midrule\noalign{}
\endhead
\bottomrule\noalign{}
\endlastfoot
Accuracy & 0.6398 \\
Kappa & 0.2736 \\
Sensitivity & 0.6437 \\
Specificity & 0.6344 \\
Valor predictivo positivo (PPV) & 0.7070 \\
Valor predictivo negativo (NPV) & 0.5651 \\
Balanced Accuracy & 0.6391 \\
No Information Rate & 0.5782 \\
P-Value (Acc \textgreater{} NIR) & 0.0000 \\
Mcnemar's Test P-Value & 0.0000 \\
\end{longtable}

\subsection{Probit}\label{probit}

\begin{longtable}[]{@{}lr@{}}
\caption{Resultados del modelo Probit (Matriz de
confusión)}\tabularnewline
\toprule\noalign{}
Métrica & Valor \\
\midrule\noalign{}
\endfirsthead
\toprule\noalign{}
Métrica & Valor \\
\midrule\noalign{}
\endhead
\bottomrule\noalign{}
\endlastfoot
Accuracy & 0.6380 \\
Kappa & 0.2699 \\
Sensibilidad & 0.6421 \\
Especificidad & 0.6323 \\
Valor predictivo positivo (PPV) & 0.7055 \\
Valor predictivo negativo (NPV) & 0.5629 \\
Balanced Accuracy & 0.6372 \\
No Information Rate & 0.5784 \\
P-Value (Acc \textgreater{} NIR) & 0.0000 \\
Mcnemar's Test P-Value & 0.0000 \\
\end{longtable}

\subsection{Naive Bayes}\label{naive-bayes-1}

\begin{longtable}[]{@{}lr@{}}
\caption{Resultados del modelo Naive Bayes (Matriz de
confusión)}\tabularnewline
\toprule\noalign{}
Métrica & Valor \\
\midrule\noalign{}
\endfirsthead
\toprule\noalign{}
Métrica & Valor \\
\midrule\noalign{}
\endhead
\bottomrule\noalign{}
\endlastfoot
Accuracy & 0.5961 \\
Kappa & 0.1735 \\
Sensibilidad & 0.5858 \\
Especificidad & 0.6218 \\
Valor predictivo positivo (PPV) & 0.7945 \\
Valor predictivo negativo (NPV) & 0.3756 \\
Balanced Accuracy & 0.6038 \\
No Information Rate & 0.7139 \\
P-Value (Acc \textgreater{} NIR) & 1.0000 \\
Mcnemar's Test P-Value & 0.0000 \\
\end{longtable}

\subsection{KNN}\label{knn}

\begin{longtable}[]{@{}lr@{}}
\caption{Resultados del modelo K-Nearest Neighbors (Matriz de
confusión)}\tabularnewline
\toprule\noalign{}
Métrica & Valor \\
\midrule\noalign{}
\endfirsthead
\toprule\noalign{}
Métrica & Valor \\
\midrule\noalign{}
\endhead
\bottomrule\noalign{}
\endlastfoot
Accuracy & 0.6114 \\
Kappa & 0.2125 \\
Sensibilidad & 0.6102 \\
Especificidad & 0.6135 \\
Valor predictivo positivo (PPV) & 0.7248 \\
Valor predictivo negativo (NPV) & 0.4854 \\
Balanced Accuracy & 0.6118 \\
No Information Rate & 0.6252 \\
P-Value (Acc \textgreater{} NIR) & 0.9624 \\
Mcnemar's Test P-Value & 0.0000 \\
\end{longtable}

\subsection{Arbol de decisión}\label{arbol-de-decisiuxf3n}

\begin{longtable}[]{@{}lrr@{}}
\caption{Comparación de desempeño: Árboles de decisión (CART
vs.~C5.0)}\tabularnewline
\toprule\noalign{}
Métrica & CART & C5.0 \\
\midrule\noalign{}
\endfirsthead
\toprule\noalign{}
Métrica & CART & C5.0 \\
\midrule\noalign{}
\endhead
\bottomrule\noalign{}
\endlastfoot
Accuracy & 0.6120 & 0.6161 \\
Kappa & 0.2208 & 0.2243 \\
Sensibilidad & 0.6286 & 0.6115 \\
Especificidad & 0.5926 & 0.6192 \\
Valor predictivo positivo (PPV) & 0.6423 & 0.5195 \\
Valor predictivo negativo (NPV) & 0.5783 & 0.7031 \\
Balanced Accuracy & 0.6106 & 0.6154 \\
No Information Rate & 0.5378 & 0.5977 \\
P-Value (Acc \textgreater{} NIR) & 0.0000 & 0.0101 \\
Mcnemar's Test P-Value & 0.2656 & 0.0000 \\
\end{longtable}

\subsection{Redes Neuronales}\label{redes-neuronales}

\begin{longtable}[]{@{}lr@{}}
\caption{Resultados del modelo Red Neuronal (nnet)}\tabularnewline
\toprule\noalign{}
Métrica & Valor \\
\midrule\noalign{}
\endfirsthead
\toprule\noalign{}
Métrica & Valor \\
\midrule\noalign{}
\endhead
\bottomrule\noalign{}
\endlastfoot
Accuracy & 0.6299 \\
Kappa & 0.2543 \\
Sensibilidad & 0.6372 \\
Especificidad & 0.6203 \\
Valor predictivo positivo (PPV) & 0.6897 \\
Valor predictivo negativo (NPV) & 0.5634 \\
Balanced Accuracy & 0.6287 \\
No Information Rate & 0.5698 \\
P-Value (Acc \textgreater{} NIR) & 0.0000 \\
Mcnemar's Test P-Value & 0.0000 \\
\end{longtable}

\begin{longtable}[]{@{}lr@{}}
\caption{Comparación de Accuracy entre modelos de
clasificación}\tabularnewline
\toprule\noalign{}
Modelo & Accuracy \\
\midrule\noalign{}
\endfirsthead
\toprule\noalign{}
Modelo & Accuracy \\
\midrule\noalign{}
\endhead
\bottomrule\noalign{}
\endlastfoot
Logit & 0.6398 \\
Probit & 0.6380 \\
Naive Bayes & 0.5961 \\
KNN & 0.6114 \\
CART & 0.6120 \\
C5.0 & 0.6161 \\
Red Neuronal & 0.6299 \\
\end{longtable}

Dado los resultados que se pueden ver en la tabla 8, se eligio el modelo
logit para realizar la función predictiva, se elaboro una función
predecir\_inseguridad() para el modelo predictivo:

\begin{verbatim}
predecir_inseguridad <- function(sexo, edad, p_indig, 
                                 aestudio, yhog, totper,
                                 s06a_02, s06a_06, 
                                 s06a_14, s06a_19, refrigerador) {
  nuevo_hogar <- data.frame(
    sexo = factor(sexo, 
                  levels = levels(bdtrain$sexo)),
    edad = edad,
    p_indig = p_indig,
    aestudio = aestudio,
    yhog = yhog,
    totper = totper,
    s06a_02 = factor(s06a_02, 
                     levels = levels(bdtrain$s06a_02)),
    s06a_06 = factor(s06a_06, 
                     levels = levels(bdtrain$s06a_06)),
    s06a_14 = s06a_14,
    s06a_19 = s06a_19,
    refrigerador = refrigerador
  )
  if (any(is.na(nuevo_hogar))) {
    print("Error: alguno de los valores ingresados 
          no coincide con los niveles esperados.")
    return(nuevo_hogar)
  }
  prob <- predict(m1, newdata = nuevo_hogar, 
                  type = "response")
  resultado <- ifelse(prob > 0.5, "SÍ sufre inseguridad 
                      alimentaria leve", 
                      "NO sufre inseguridad alimentaria leve")
  list(probabilidad = round(prob, 3), resultado = resultado)
}
\end{verbatim}

Para poder utilizar la función y predecir si existe inseguridad
alimentaria en un hogar, dadas 11 variables, de las cuales 4 son del
jefe/a de hogar y el resto son caracteristicas de la vivienda, se debe
llenar el formulario de la siguiente forma:

\begin{verbatim}
predecir_inseguridad(
  sexo = "1. Hombre",
  edad = 40,
  p_indig = 0,
  aestudio = 8,
  yhog = 2000,
  totper = 4,
  s06a_02 = "3. ¿Alquilada?",
  s06a_06 = "5. CEMENTO",
  s06a_14 = 1,
  s06a_19 = 1,
  refrigerador = 1
)
\end{verbatim}

en la que las variables categoricas nóminales deben ser llenadas con
alguna de las siguientes opciones:

\begin{itemize}
\tightlist
\item
  sexo: 1. Hombre, 2. Mujer
\item
  s06a\_02: 1. ¿Propia y totalmente pagada?, 2. ¿Propia y la están
  pagando?, 3. ¿Alquilada? 4. ¿En contrato Mixto (alquiler y
  anticretico)?, 5. ¿En contrato anticretico?, 6. ¿Cedida por
  servicios?, 7. ¿Prestada por parientes o amigos? u 8.
  ¿Otra?(Especifique)
\item
  s06a\_06: 1. TIERRA, 2. TABLÓN DE MADERA, 3. MACHIHEMBRE/PARQUET, 4.
  PISO FLOTANTE, 5. CEMENTO, 6. MOSAICO/BALDOSAS/CERÁMICA, 7. LADRILLO u
  8. OTRO (Especifique)
\end{itemize}

Una vez llenado correctamente y ejecutado la función, se obtendra un
resultado como el siguiente:

\begin{verbatim}
# Ejemplo de salida de la función predecir_inseguridad
$probabilidad
    1 
0.633 

$resultado
                                      1 
"SÍ sufre inseguridad alimentaria leve" 
\end{verbatim}

Podemos apreciar que el modelo esta funcionando correctamente, las
caracteristicas de la vivienda de ejemplo son un hogar con 4 integrantes
y un ingreso promedio de Bs. 2000, lo que puede ya darnos un gran
indicio de que este hogar sufre inseguridad alimentaria por falta de
recursos. Además podemos ver las otras variables como por ejemplo que la
vivienda es alquilada, por lo tanto gran parte del ingreso del hogar se
destina a alquiler, restando recursos disponibles para una buena
alimentación.

\section{Conclusiones y
recomendaciones}\label{conclusiones-y-recomendaciones}

Los resultados de este estudio muestran que es posible predecir la
probabilidad de que un hogar boliviano experimente inseguridad
alimentaria leve a partir de características sociodemográficas y
materiales observables, utilizando modelos de clasificación supervisada.
El modelo logit, que obtuvo el mejor desempeño entre las técnicas
comparadas, alcanzó una precisión de aproximadamente 63.9\%, lo cual
sugiere una capacidad predictiva moderada.

Si bien este nivel de exactitud no es el mejor (63.9\%), especialmente
considerando que se trata de un problema socialmente sensible, es
importante destacar que los modelos desarrollados no pretenden
reemplazar mecanismos formales de diagnóstico, sino ofrecer una
herramienta complementaria para focalizar intervenciones. La inseguridad
alimentaria leve, al estar basada en la experiencia subjetiva de
preocupación por la alimentación, puede estar influida por factores no
directamente observables en las encuestas. Esto podría explicar parte
del margen de error del modelo.

A pesar de esta limitación, el enfoque predictivo basado en minería de
datos tiene alto valor aplicado. Puede servir como un insumo útil para
el diseño de políticas públicas focalizadas, especialmente en programas
de transferencias, subsidios alimentarios o asistencia nutricional.

Se recomienda ampliar el conjunto de variables predictoras incluyendo
información sobre empleo, gasto alimentario o acceso a redes de apoyo,
aplicar otras técnicas como random forest o boosting, que podrían
mejorar el rendimiento predictivo, utilizar métricas adicionales como el
AUC-ROC para evaluar el balance entre sensibilidad y especificidad.

En conclusión, el presente trabajo demuestra que, incluso con
exactitudes intermedias, los modelos de clasificación pueden aportar
evidencia valiosa para combatir la inseguridad alimentaria desde un
enfoque preventivo y focalizado, orientando recursos limitados hacia los
hogares con mayor riesgo.

\newpage

`

\bibliographystyle{sageh}
\bibliography{bibfile.bib}


\end{document}
